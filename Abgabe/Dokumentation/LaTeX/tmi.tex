\documentclass[journal,twoside,web]{ieeecolor}
\usepackage{tmi}
\usepackage{amsmath,amssymb,amsfonts}
\usepackage{algorithmic}
\usepackage{graphicx}
\graphicspath{{images}}
\usepackage{textcomp}
\def\BibTeX{{\rm B\kern-.05em{\sc i\kern-.025em b}\kern-.08em
    T\kern-.1667em\lower.7ex\hbox{E}\kern-.125emX}}
\markboth{\journalname, VOL. XX, NO. XX, XXXX 2020}
{Author \MakeLowercase{\textit{et al.}}: Preparation of Papers for IEEE TRANSACTIONS ON MEDICAL IMAGING}


\begin{document}
\title{Verarbeitung von Gesichtsaufnahmen zur Genderklassifikation als Anwendung neuronaler Netze}
\author{Gruppe: Niklas Herrhofer, Celine Schneider, Andreas Braig\IEEEmembership{TEL22AT1},
\thanks{This paragraph of the first footnote will contain the date on which
you submitted your paper for review. It will also contain support information,
including sponsor and financial support acknowledgment. For example, 
``This work was supported in part by the U.S. Department of Commerce under Grant BS123456.'' }
\thanks{The next few paragraphs should contain the authors' current affiliations,
including current address and e-mail. For example, F. A. Author is with the
National Institute of Standards and Technology, Boulder, CO 80305 USA (e-mail:author@boulder.nist.gov). }
\thanks{S. B. Author, Jr., was with Rice University, Houston, TX 77005 USA.
He is now with the Department of Physics, Colorado State University,
Fort Collins, CO 80523 USA (e-mail: author@lamar.colostate.edu).}
\thanks{T. C. Author is with the Electrical Engineering Department,
University of Colorado, Boulder, CO 80309 USA, on leave from the National
Research Institute for Metals, Tsukuba, Japan (e-mail: author@nrim.go.jp).}}

\maketitle

\begin{abstract}
    Brauchen wir echt ein Abstract? 
\end{abstract}


\begin{IEEEkeywords}
    Neuronales Netz, Bildsegmentierung, Convolutional Neural Network, 
\end{IEEEkeywords}

\section{Problemstellung und Ziel dieser Arbeit}
\label{sec:introduction}
\IEEEPARstart{D}{iese} Arbeit befasst sich Mit der Verarbeitung und Klassifikation einzelner Datenpunkte in einem Datensatz.
Die hierbei verwendeten Daten sind Gesichtsaufnahmen verschiedener Personen verschiedenen Alters. 

Die erste Teilaufgabe besteht in der Segmentierung und Verarbeitung dieses bereitgestellten Datensatzes, um die Gesichtselemente einheitlich im Bild zu positionieren. 
Augen und Mund sollen hierbei immer ein gleichschenkliges Dreieck an einer festen Position im Bild bilden. 

Die zweite Teilaufgabe befasst sich mit dem Klassifikationsproblem. 
Der verarbeitete Datensatz wird in ein Convolutional Neural Network geladen und das Geschlecht der abgebildeten Person klassifiziert. 
Hierbei soll das Netz eine binäre Klassifikation zwischen Männlich (0) und Weiblich (1) durchführen.

Für die Lösung dieser Aufgabe wurde die Programmiersprache Python mit den wesentlichen Bibliotheken "OpenCV", "numpy" und "Pytorch" verwendet. 

\section{Stand der Technik}
In diesem Kapitel wird der aktuelle Stand der Technik zur Bildsegmentierung und Klassifizierung Diskutiert.

\subsection{Python}


\subsection{Klassifikationsprobleme}

\begin{figure}[!t]
    \centerline{\includegraphics[width=\columnwidth]{Andi/binaere_klassifikation.png}}
    \caption{Magnetization as a function of applied field.
    It is good practice to explain the significance of the figure in the caption.}
    \label{fig:fig1}
\end{figure}



\section{Der Datensatz}
In diesem Kapitel wird der verwendete Datensatz vorgestellt. Es soll nachvollziehbar sein welche Herausforderungen bei der Arbeit mit diesen Daten auftreten. 


\smallskip\noindent
\begin{small}
\begin{tabular}{l}
\verb+\+\texttt{documentclass[journal,twoside,web]\{ieeecolor\}}\\
\verb+\+\texttt{usepackage\{\textit{Journal\_Name}\}}
\end{tabular}
\end{small}

\section{Units}

\begin{figure}[!t]
\centerline{\includegraphics[width=\columnwidth]{Architektur.png}}
\caption{Darstellung der Programmarchitektur in Form eines Blockschaltbildes}
\label{fig1}
\end{figure}


\section{Guidelines for Graphics Preparation and Submission}
\label{sec:guidelines}

\subsection{Types of Graphics}

\subsubsection{Tables}
{Data charts which are typically black and white, but sometimes include 
color.}

\begin{figure}[!t]
    \centerline{\includegraphics[width=\columnwidth]{Andi/Loss_Acc_Gewinner.png}}
    \caption{Trainingslog des CNN Modells, welches im Test die besten Ergebnisse Erzielt}
    \label{fig:fig1}
\end{figure}

\section{Conclusion}

\appendices

\section*{Appendix and the Use of Supplemental Files}


\section*{Acknowledgment}


\end{document}
